%!TEX root = ../thesis.tex
\chapter{Conclusions and Future Work}
\label{chap:conclusions}

This capstone project builds a Python platform that can be used to test quantitative trading models in a network setting under client/server infrastructure with performance analysis displayed on web dashboard. From our backtesting and simulated trading results, we see that the performance of simulated trading is much worse than that of backtesting, although backtesting shows that our trading model has a 2.5 Sharpe ratio and 25\% annual return. It concludes that we should be careful in real-time trading even with a profitable model shown in backtesting and we always want to do paper trading before real trading. 

This python platform is a good tool for testing trading models, alerting for potential losses. However, there are limitations. Our assumptions of order book are simple and our simulated trading is not equivalent to paper trading. Paper trading is to trade based on real-time order books with no real money evolved. Our simulated trading has order book that is simulated from historical data and is fixed for each day. Our order book is too simple as compared to a real order book and is lack of real market dynamics. We would need tick level data to actually simulate a real market, for our first improvement that can be done to this project. Furthermore, we can improve our order matching algorithm and allow for market orders. We always place order at the best bid and offer of one order book of that day and it is always filled at this price. While in the real market, the limit order is fulfilled at our price and better, possibly split into different trades. It also has market order that takes many prices to get as much quantity filled as possible. For our pairs trading model, market orders are more reasonable because we are doing dollar neutral strategy and our purpose is to get all quantity of orders filled.

There are also ways to improve our trading model. First, we have many hyperparameters in our trading model, including training and testing time periods, significance level in adjusted dickey fuller test, number of standard deviations and moving average period in bollinger band construction, number of principle components and epsilon in PCA. These hyperparameters can be optimized through cross validation. Second, we can do kalman filter instead of one simple ordinary least square. Third, we should set up stop loss level to prevent losses as in our simulated trading. Fourth, we should also include transaction costs in our model.